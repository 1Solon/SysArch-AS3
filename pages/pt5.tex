\section{Internet of Things and Wearable Technologies in Health Monitoring}
The Internet of Things refers to the network of physical objects embedded with
sensors and other technologies to connect and exchanging information with other
devices and systems over the internet. These objects, or "things," can range
from ordinary household items to advanced industrial tools. In the context of
health monitoring, IoT technology enhances the functionality and effectiveness
of wearable devices.

\subsection{Enhanced Data Collection  and Integration}
IoT-enabled wearables can collect a broad array of health information in
real-time, from heart rate and blood pressure to more complex metrics like blood
oxygen level and electrocardiogram readings. These devices incorporated within
the AT passport system, can transfer collected information to smartphones or
directly to centralized servers for processing. For example, continuous glucose
monitor devices that use IoT connectivity can transmit real-time glucose levels
to a patient's smartphone, which processes the information to offer insights and
send alerts if intervention is needed.

\subsection{Proactive Health Management}
The integration of IoT in wearables facilitates not just the monitoring but also
the proactive management of health conditions.\cite{huhn2022impact} By
continuously analyzing the information collected, the organization can identify
possible health issues before they get critical. For instance, IoT-enabled
wearables can detect irregular heartbeat indicative of atrial fibrillation and
automatically alert healthcare providers, enabling early treatment and
potentially preventing severe health events.\cite{huhn2022impact}

\subsection{Personalized Healthcare Experiences}
IoT technology contributes to personalized medicine by allowing wearables to
conform to the individual's particular health needs. Based on the analysis of
accumulated data, wearables can indicate lifestyle changes, medication
adjustments, or recommend consultations with healthcare providers. This level of
personalization ensures that each user receives care that is tailored to their
unique health profile, making health interventions more effective.\cite{jin2019review}

\subsection{Challenges and Considerations}
Despite the numerous benefits, the integration of IoT in health monitoring
systems raises important privacy and security concerns. The transmission of
sensitive health information over a network exposes users to possible data
breaches and unauthorized access. For example, if protection measures are not
adequately enforced, personal health data could be intercepted during
transmission from wearables to servers. Moreover, ensuring the accuracy and
reliability of IoT devices remains crucial, as faulty data could lead to
improper health interventions. Regulatory compliance, particularly concerning
data security standards like GDPR, must be strictly adhered to, to protect user
privacy and maintain confidence in the technology. While IoT technology enhances
the capabilities of wearable technology in health monitoring, careful
consideration must be given to privacy and security to fully realize its
benefits in systems like the AT passport.