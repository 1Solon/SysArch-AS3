\section{System Performance}

The proposed architecture leverages smartphones as edge compute nodes, which
significantly enhance system performance through these key aspects:

\begin{itemize}
    \item \textbf{Reduced Latency:} By processing information locally on
          smartphones, the system minimizes the need to send large volumes of
          sensitive information to the central server. This local preprocessing
          speed up response time as only relevant information is forwarded for
          further processing. \cite{ning2020mobile}
    \item \textbf{Load Distribution:} Incorporating a load balancer ensures
          efficient distribution of information across multiple servers
          instances, preventing any one server from becoming a bottleneck and
          enhancing overall response time and information handling capabilities.
    \item \textbf{Optimized Data Handling:} Separating the process and storage
          functionalities allow each to be optimized for their particular tasks,
          allowing for fast processing for real-time analytics and efficient
          storage for information persistence—improving both speed and
          efficiency. \cite{liu2024integrating}
\end{itemize}

\section{System Reliability}

Several mechanisms are employed within the architecture to guarantee high
reliability:

\begin{itemize}
    \item \textbf{Fault Tolerance:} The use of a load balancer improves system
          reliability by rerouting traffic away from servers that are failing or
          are overburdened, maintaining continuous service and allowing for
          maintenance without downtime.
    \item \textbf{Data Redundancy:} Configuring the database system for
          redundancy protects against data loss in the case of hardware failure
          and allows for recovery without data corruption.
    \item \textbf{Robust Security Measures:} Implementing security features such
          as end-to-end encryption, strong authentication, and rigorous access
          control prevents unauthorized access and data breaches, enhancing
          system reliability. \cite{pool2024systematic}
\end{itemize}

\section{Scalability of Data Storage Requirements}
The scalability of the architecture is addressed through various features:

\begin{itemize}
    \item \textbf{Modular Design:} The architecture's modular design allows for
          the addition of more smartphones as edge client and server instances
          as needed. This horizontal scalability supports an increased amount of
          users and high information processing demands.
    \item \textbf{Cloud Integration Possibility:} Although the current setup is
          non-cloud, the architecture is designed for easy integration with
          cloud services if needed, providing almost unlimited scalability by
          leveraging cloud resources for additional processing and storage
          capacity.
    \item \textbf{Efficient Data Management:} Integrating a Document Management
          A system with a database ensures effective organization and management
          of data, supporting scalability with features like indexing and fast
          recovery capabilities as data volume grows.
\end{itemize}