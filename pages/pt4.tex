\section{Negative Examples}
\subsection{Data Breaches and Unauthorized Access}
Cloud systems are open to a broader attack surface due to their internet
accessibility. For an AT passport system handling sensitive health data, this
exposure increases the chance of data breaches. For example, in 2019, a major
cloud service provider experienced a significant breach affecting hundreds of
millions of personal records. In the context of health monitoring, such a breach
could reveal extremely sensitive health information, and lead to serious privacy
violations and possible abuse of personal health data.

\subsection{Compliance and Data Sovereignty Issues}
Cloud services frequently require data centres located in multiple
jurisdictions, which can complicate compliance with stringent health data
protection regulations such as HIPAA or GDPR. If information from a European
citizen is stored outside the EU, it may not be protected under GDPR, leading to
possible legal and compliance risks. This scenario can result in fines and
sanctions against healthcare providers or entities managing the AT passport
system, alongside the danger of losing public trust.

\section{Positive Example}
\subsection{Scalability and Accessibility}
Despite security concerns, cloud services provide unmatched scalability and
accessibility, which is vital for the efficient deployment of an AT passport
system. Cloud platforms can dynamically allocate resources to manage increasing
data and user access requests without significant upfront investment in physical
infrastructure. For example, during the COVID-19 pandemic, healthcare provider
rapidly scale up their telehealth service using cloud solutions to match the
sudden surge in demand for remote healthcare, ensure that patient continue to
have necessary health monitoring.

\section{Conclusion} While the benefits of cloud services, especially their
scalability and accessibility, are undeniable, the threat to privacy and
security, particularly concerning sensitive health data, is important and often
outweigh the benefits. Data breaches and compliance issues not only present
danger to individual privacy but also threaten the legal and operational
standing of entities managing health data. Decisions to apply cloud services in
the setting of health monitoring systems must be accompanied by robust security
measures, rigorous compliance checks, and transparent information handling
practices to mitigate these risks effectively.